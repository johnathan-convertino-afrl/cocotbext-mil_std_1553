\begin{titlepage}
  \begin{center}

  {\Huge COCOTBEXT\_mil-std-1553}

  \vspace{25mm}

  \includegraphics[width=0.90\textwidth,height=\textheight,keepaspectratio]{img/AFRL.png}

  \vspace{25mm}

  \today

  \vspace{15mm}

  {\Large Jay Convertino}

  \end{center}
\end{titlepage}

\tableofcontents

\newpage

\section{Usage}

\subsection{Introduction}

\par
Standard AXIS FIFO with multiple options. The FIFO uses a AXIS interface for data in and out.
It also emulates the Xilinx AXIS FIFO bugs and all. This is NOT dependent on Xilinx FPGA's and
can be used on any FPGA supporting the Verilog block ram style primitive.

\subsection{Dependencies}

\par
The following are the dependencies of the cores.

\begin{itemize}
  \item iverilog (simulation)
  \item cocotb (simulation)
\end{itemize}

\subsection{In a Project}
\par
Simply use this core between a sink and source devices. This buffer data from one bus to another. Check the code to see if others will work correctly.

\section{Architecture}

\par
This AXIS FIFO is made for two modules. They are the FIFO, and AXIS FIFO control. The combination of these two provide the AXIS FIFO module.
Having it made this way allows for future modules to be customized and brought in to change the FIFO's behavior. The current modules emulate the Xilinx
AXIS FIFO IP core availble in Vivado 2018 and up.

\par
AXIS FIFO control is the heart of the core when it comes to how it responds. The logic in the core is designed to emulate the Xilinx AXIS FIFO IP.

Please see \ref{Code Documentation} for more information.

\section{Building}

\par
The AXIS FIFO core is written in Verilog 2001. They should synthesize in any modern FPGA software. The core comes as a fusesoc packaged core and can be
included in any other core. Be sure to make sure you have meet the dependencies listed in the previous section.

\subsection{Directory Guide}

\par
Below highlights important folders from the root of the directory.

\begin{enumerate}
  \item \textbf{docs} Contains all documentation related to this project.
    \begin{itemize}
      \item \textbf{manual} Contains user manual and github page that are generated from the latex sources.
    \end{itemize}
  \item \textbf{src} Contains source files for the core
  \item \textbf{tb} Contains test bench files for iverilog and cocotb
\end{enumerate}

\newpage

\section{Simulation}
\par
There are a few different simulations that can be run for this core. All currently use iVerilog (icarus) to run. The first is iverilog, which
uses verilog only for the simulations. The other is cocotb. This does a unit test approach to the testing and gives a list of tests that pass
or fail.

\subsection{iverilog}
\par
All simulation targets that do NOT have cocotb in the name use a verilog test bench with verilog stimulus components. These all read in a file
and then write a file that has been processed by the FIFO. Then the input and output file are compared with a MD5 sum to check that they
match. If they do not match then the test has failed. All of these tests provide fst output files for viewing the waveform in the there
target build folder.

\subsection{cocotb}
\par
To use the cocotb tests you must install the following python libraries.
\begin{lstlisting}[language=bash]
  $ pip install cocotb
  $ pip install cocotbext-axi
\end{lstlisting}

Then you must use the cocotb sim target. In this case it is sim\_cocotb. This target can be run with various bus and fifo parameters.
\begin{lstlisting}[language=bash]
  $ fusesoc run --target sim_cocotb AFRL:buffer:axis_fifo:1.0.0 --BUS_WIDTH=8 --FIFO_DEPTH=32
\end{lstlisting}

The following is an example command to run through various parameters without typing them one by one.
\begin{lstlisting}[language=bash]
  $ for i in {1..32}; do sleep 5; export RY=$(($RANDOM%32+1)); fusesoc run --target sim_cocotb AFRL:buffer:axis_fifo:1.0.0 --BUS_WIDTH=$i --FIFO_DEPTH=$RY; echo "BUS WIDTH:" $i "FIFO DEPTH:" $RY; done
\end{lstlisting}
\newpage

\section{Code Documentation} \label{Code Documentation}

\par
Natural docs is used to generate documentation for this project. The next lists the following sections.

\begin{itemize}
\item \textbf{AXIS\_FIFO} will buffer data from input to output.\\
\item \textbf{AXIS\_FIFO\_CONTROL} emulates the Xilinx FIFO IP interface and its behavior.\\
\item \textbf{tb\_axis} Verilog test bench.\\
\item \textbf{tb\_cocotb verilog} Verilog test bench base for cocotb.\\
\item \textbf{tb\_cocotb python} cocotb unit test functions.\\
\end{itemize}

